% Options for packages loaded elsewhere
\PassOptionsToPackage{unicode}{hyperref}
\PassOptionsToPackage{hyphens}{url}
%
\documentclass[
]{article}
\usepackage{amsmath,amssymb}
\usepackage{iftex}
\ifPDFTeX
  \usepackage[T1]{fontenc}
  \usepackage[utf8]{inputenc}
  \usepackage{textcomp} % provide euro and other symbols
\else % if luatex or xetex
  \usepackage{unicode-math} % this also loads fontspec
  \defaultfontfeatures{Scale=MatchLowercase}
  \defaultfontfeatures[\rmfamily]{Ligatures=TeX,Scale=1}
\fi
\usepackage{lmodern}
\ifPDFTeX\else
  % xetex/luatex font selection
\fi
% Use upquote if available, for straight quotes in verbatim environments
\IfFileExists{upquote.sty}{\usepackage{upquote}}{}
\IfFileExists{microtype.sty}{% use microtype if available
  \usepackage[]{microtype}
  \UseMicrotypeSet[protrusion]{basicmath} % disable protrusion for tt fonts
}{}
\makeatletter
\@ifundefined{KOMAClassName}{% if non-KOMA class
  \IfFileExists{parskip.sty}{%
    \usepackage{parskip}
  }{% else
    \setlength{\parindent}{0pt}
    \setlength{\parskip}{6pt plus 2pt minus 1pt}}
}{% if KOMA class
  \KOMAoptions{parskip=half}}
\makeatother
\usepackage{xcolor}
\usepackage[margin=1in]{geometry}
\usepackage{color}
\usepackage{fancyvrb}
\newcommand{\VerbBar}{|}
\newcommand{\VERB}{\Verb[commandchars=\\\{\}]}
\DefineVerbatimEnvironment{Highlighting}{Verbatim}{commandchars=\\\{\}}
% Add ',fontsize=\small' for more characters per line
\usepackage{framed}
\definecolor{shadecolor}{RGB}{248,248,248}
\newenvironment{Shaded}{\begin{snugshade}}{\end{snugshade}}
\newcommand{\AlertTok}[1]{\textcolor[rgb]{0.94,0.16,0.16}{#1}}
\newcommand{\AnnotationTok}[1]{\textcolor[rgb]{0.56,0.35,0.01}{\textbf{\textit{#1}}}}
\newcommand{\AttributeTok}[1]{\textcolor[rgb]{0.13,0.29,0.53}{#1}}
\newcommand{\BaseNTok}[1]{\textcolor[rgb]{0.00,0.00,0.81}{#1}}
\newcommand{\BuiltInTok}[1]{#1}
\newcommand{\CharTok}[1]{\textcolor[rgb]{0.31,0.60,0.02}{#1}}
\newcommand{\CommentTok}[1]{\textcolor[rgb]{0.56,0.35,0.01}{\textit{#1}}}
\newcommand{\CommentVarTok}[1]{\textcolor[rgb]{0.56,0.35,0.01}{\textbf{\textit{#1}}}}
\newcommand{\ConstantTok}[1]{\textcolor[rgb]{0.56,0.35,0.01}{#1}}
\newcommand{\ControlFlowTok}[1]{\textcolor[rgb]{0.13,0.29,0.53}{\textbf{#1}}}
\newcommand{\DataTypeTok}[1]{\textcolor[rgb]{0.13,0.29,0.53}{#1}}
\newcommand{\DecValTok}[1]{\textcolor[rgb]{0.00,0.00,0.81}{#1}}
\newcommand{\DocumentationTok}[1]{\textcolor[rgb]{0.56,0.35,0.01}{\textbf{\textit{#1}}}}
\newcommand{\ErrorTok}[1]{\textcolor[rgb]{0.64,0.00,0.00}{\textbf{#1}}}
\newcommand{\ExtensionTok}[1]{#1}
\newcommand{\FloatTok}[1]{\textcolor[rgb]{0.00,0.00,0.81}{#1}}
\newcommand{\FunctionTok}[1]{\textcolor[rgb]{0.13,0.29,0.53}{\textbf{#1}}}
\newcommand{\ImportTok}[1]{#1}
\newcommand{\InformationTok}[1]{\textcolor[rgb]{0.56,0.35,0.01}{\textbf{\textit{#1}}}}
\newcommand{\KeywordTok}[1]{\textcolor[rgb]{0.13,0.29,0.53}{\textbf{#1}}}
\newcommand{\NormalTok}[1]{#1}
\newcommand{\OperatorTok}[1]{\textcolor[rgb]{0.81,0.36,0.00}{\textbf{#1}}}
\newcommand{\OtherTok}[1]{\textcolor[rgb]{0.56,0.35,0.01}{#1}}
\newcommand{\PreprocessorTok}[1]{\textcolor[rgb]{0.56,0.35,0.01}{\textit{#1}}}
\newcommand{\RegionMarkerTok}[1]{#1}
\newcommand{\SpecialCharTok}[1]{\textcolor[rgb]{0.81,0.36,0.00}{\textbf{#1}}}
\newcommand{\SpecialStringTok}[1]{\textcolor[rgb]{0.31,0.60,0.02}{#1}}
\newcommand{\StringTok}[1]{\textcolor[rgb]{0.31,0.60,0.02}{#1}}
\newcommand{\VariableTok}[1]{\textcolor[rgb]{0.00,0.00,0.00}{#1}}
\newcommand{\VerbatimStringTok}[1]{\textcolor[rgb]{0.31,0.60,0.02}{#1}}
\newcommand{\WarningTok}[1]{\textcolor[rgb]{0.56,0.35,0.01}{\textbf{\textit{#1}}}}
\usepackage{longtable,booktabs,array}
\usepackage{calc} % for calculating minipage widths
% Correct order of tables after \paragraph or \subparagraph
\usepackage{etoolbox}
\makeatletter
\patchcmd\longtable{\par}{\if@noskipsec\mbox{}\fi\par}{}{}
\makeatother
% Allow footnotes in longtable head/foot
\IfFileExists{footnotehyper.sty}{\usepackage{footnotehyper}}{\usepackage{footnote}}
\makesavenoteenv{longtable}
\usepackage{graphicx}
\makeatletter
\def\maxwidth{\ifdim\Gin@nat@width>\linewidth\linewidth\else\Gin@nat@width\fi}
\def\maxheight{\ifdim\Gin@nat@height>\textheight\textheight\else\Gin@nat@height\fi}
\makeatother
% Scale images if necessary, so that they will not overflow the page
% margins by default, and it is still possible to overwrite the defaults
% using explicit options in \includegraphics[width, height, ...]{}
\setkeys{Gin}{width=\maxwidth,height=\maxheight,keepaspectratio}
% Set default figure placement to htbp
\makeatletter
\def\fps@figure{htbp}
\makeatother
\setlength{\emergencystretch}{3em} % prevent overfull lines
\providecommand{\tightlist}{%
  \setlength{\itemsep}{0pt}\setlength{\parskip}{0pt}}
\setcounter{secnumdepth}{-\maxdimen} % remove section numbering
\ifLuaTeX
  \usepackage{selnolig}  % disable illegal ligatures
\fi
\IfFileExists{bookmark.sty}{\usepackage{bookmark}}{\usepackage{hyperref}}
\IfFileExists{xurl.sty}{\usepackage{xurl}}{} % add URL line breaks if available
\urlstyle{same}
\hypersetup{
  pdftitle={Data Science Project EDA},
  pdfauthor={Kaylee Davis},
  hidelinks,
  pdfcreator={LaTeX via pandoc}}

\title{Data Science Project EDA}
\author{Kaylee Davis}
\date{2023-12-12}

\begin{document}
\maketitle

\hypertarget{introduction}{%
\section{Introduction}\label{introduction}}

As societies grow more reliant on technology and machine learning,
understanding public sentiment is crucial for responsible development
and implementation. This report delves into the dynamics of technology
adoption and recommendation behaviors across diverse demographic groups.
It focuses on understanding how different factors, specifically, age and
household income, influence individuals' appraisals of technology,
focusing on concerns toward technology In general, insights regarding
population-level perceptions of technology are crucial for stakeholders
such as marketers, policymakers, and tech developers, highlighting the
importance of tailoring strategies to effectively reach specific
demographic segments. By pinpointing various predictors of adoption
behavior and technology concern, the dataset offers valuable guidance
for developing targeted, inclusive policies and educational initiatives
to foster digital literacy and bridge adoption gaps.

\hypertarget{research-questions}{%
\subsubsection{Research Questions}\label{research-questions}}

Q1: Whether age-group affects appraisals of technology Q2: Whether
income-level negatively affects an individuals appraisals of technology
(lower = more neg)

\hypertarget{method}{%
\section{Method}\label{method}}

The study utilized a quantitative methodology, analyzing data from an
online dataset acquired from kaggle. The dataset examined technology
use, concerns, influences, barriers to adoption, and likelihood of
recommending emerging technologies. This dataset is utilized to
determine if age or income level might predict skepticism toward
technology. However, after conducting statistical tests and using
descriptive statistics to visualize the relationship, there was no
significant effect for these predictions.

I used the Technology Perception Survey dataset to investigate these
questions. This dataset is relevant due to its focus on demographic
variables (age, socioeconomic status, etc.) and attitudes towards
technology (barriers to adoption, likeliness to adopt, etc.) The data
was obtained from the Kaggle website and generated by an individual
named Calvin White.

The dataset examined technology use, concerns, influences, barriers to
adoption, and likelihood of recommending emerging technologies. This
dataset is utilized to determine if age or income level might predict
skepticism toward technology. The group surveyed were individuals living
in India. Their responses provided insights into their usage, concerns,
influence, barriers, and likelihood to adopt and recommend technologies
as the original intended use of the dataset is to understand consumer
behavior and gain insights into how different demographic groups
perceive and interact with technology. The comprehensiveness and focus
on technology perception provide a rich basis for analysis. While the
dataset seems to be collected without apparent bias from the researcher,
a major limitation of the study is that the sample is largely from
India, making it unrepresentative of global consumer household
tendencies. Additionally, there is an overrepresentation of male
participants in the sample. This lack of representation could lead to
potential blind spots in understanding how different groups perceive and
interact with technology.

Necessary Transformations: I created additional variables that
consolidate some of the categories to streamline the data, making it
more conducive to deeper analysis. In the report provided by the author,
this is a similar strategy they took when analyzing the data. I used
descriptive statistics and data visualization to get an overarching view
of the trends and interactions between variables of interest

\hypertarget{results}{%
\section{Results}\label{results}}

Figure 3 displays the age and gender distribution of participants. This
distribution provides insights into the gender representation within the
study and enables the examination of potential gender-related
differences in technology adoption behavior. Additionally, Figure 3
shows that the sample consists of individuals from various age groups,
allowing for a comprehensive analysis of technology adoption behaviors
across different life stages. However, as noted before, the sample is
predominantly male, which makes it harder to generalize findings.

To understand if age-group affects appraisals of technology, Figure 4
shows the correlation between age and broad concerns of technology.
`TotalTechConcern' related to broad concerns toward technology, such as
complexity of use, health impact, and resistance to change. The original
dataset separated these into columns that began with ``Risk\_'', and the
author suggested summing these columns to get a total concern score.

In Figure 4, statistical analysis suggests no significant difference in
TotalTechConcern among age groups (p \textgreater{} 0.05). This suggests
that age alone is not be a strong determining factor in individuals'
TotalTechConcern. Other factors, such as technological familiarity,
personal experiences, and individual preferences, may have a more
significant influence on individuals' perceived concerns or risks
associated with technology.

Figure 5 aims to answer whether income level might negatively affect an
individual's appraisals of technology. There was no statistical
correlation found, this implies that household income alone may not be a
strong determining factor of their concerns toward technology.

\hypertarget{setup}{%
\subsubsection{Setup}\label{setup}}

\begin{Shaded}
\begin{Highlighting}[]
\CommentTok{\# Using renv restore function to get dependencies }
\FunctionTok{library}\NormalTok{(renv)}
\FunctionTok{restore}\NormalTok{()}

\CommentTok{\#Load required packages}
\FunctionTok{library}\NormalTok{(ggplot2)}
\FunctionTok{library}\NormalTok{(dplyr)}
\FunctionTok{library}\NormalTok{(readr)}
\FunctionTok{library}\NormalTok{(here)}
\FunctionTok{library}\NormalTok{(skimr)}
\FunctionTok{library}\NormalTok{(here)}
\FunctionTok{library}\NormalTok{(tidyverse)}
\FunctionTok{library}\NormalTok{(knitr)}
\FunctionTok{library}\NormalTok{(tinytex)}
\end{Highlighting}
\end{Shaded}

\hypertarget{read-in-data}{%
\subsubsection{Read in data}\label{read-in-data}}

\begin{Shaded}
\begin{Highlighting}[]
\FunctionTok{library}\NormalTok{(readr)}
\FunctionTok{library}\NormalTok{(here)}
\end{Highlighting}
\end{Shaded}

\begin{verbatim}
## here() starts at C:/Users/kayle/OneDrive/Desktop/Cogs-212-EDA
\end{verbatim}

\begin{Shaded}
\begin{Highlighting}[]
\NormalTok{SurveyData }\OtherTok{\textless{}{-}} \FunctionTok{read\_csv}\NormalTok{(}\FunctionTok{here}\NormalTok{(}\StringTok{"data"}\NormalTok{,}\StringTok{"SurveyData.csv"}\NormalTok{), }\AttributeTok{show\_col\_types =} \ConstantTok{FALSE}\NormalTok{)}
\end{Highlighting}
\end{Shaded}

\hypertarget{check-the-packaging}{%
\subsubsection{Check the packaging}\label{check-the-packaging}}

\begin{Shaded}
\begin{Highlighting}[]
\FunctionTok{library}\NormalTok{(skimr)}

\FunctionTok{skim}\NormalTok{(SurveyData)}
\end{Highlighting}
\end{Shaded}

\begin{longtable}[]{@{}ll@{}}
\caption{Data summary}\tabularnewline
\toprule\noalign{}
\endfirsthead
\endhead
\bottomrule\noalign{}
\endlastfoot
Name & SurveyData \\
Number of rows & 107 \\
Number of columns & 36 \\
\_\_\_\_\_\_\_\_\_\_\_\_\_\_\_\_\_\_\_\_\_\_\_ & \\
Column type frequency: & \\
character & 3 \\
numeric & 33 \\
\_\_\_\_\_\_\_\_\_\_\_\_\_\_\_\_\_\_\_\_\_\_\_\_ & \\
Group variables & None \\
\end{longtable}

\textbf{Variable type: character}

\begin{longtable}[]{@{}
  >{\raggedright\arraybackslash}p{(\columnwidth - 14\tabcolsep) * \real{0.2055}}
  >{\raggedleft\arraybackslash}p{(\columnwidth - 14\tabcolsep) * \real{0.1370}}
  >{\raggedleft\arraybackslash}p{(\columnwidth - 14\tabcolsep) * \real{0.1918}}
  >{\raggedleft\arraybackslash}p{(\columnwidth - 14\tabcolsep) * \real{0.0548}}
  >{\raggedleft\arraybackslash}p{(\columnwidth - 14\tabcolsep) * \real{0.0548}}
  >{\raggedleft\arraybackslash}p{(\columnwidth - 14\tabcolsep) * \real{0.0822}}
  >{\raggedleft\arraybackslash}p{(\columnwidth - 14\tabcolsep) * \real{0.1233}}
  >{\raggedleft\arraybackslash}p{(\columnwidth - 14\tabcolsep) * \real{0.1507}}@{}}
\toprule\noalign{}
\begin{minipage}[b]{\linewidth}\raggedright
skim\_variable
\end{minipage} & \begin{minipage}[b]{\linewidth}\raggedleft
n\_missing
\end{minipage} & \begin{minipage}[b]{\linewidth}\raggedleft
complete\_rate
\end{minipage} & \begin{minipage}[b]{\linewidth}\raggedleft
min
\end{minipage} & \begin{minipage}[b]{\linewidth}\raggedleft
max
\end{minipage} & \begin{minipage}[b]{\linewidth}\raggedleft
empty
\end{minipage} & \begin{minipage}[b]{\linewidth}\raggedleft
n\_unique
\end{minipage} & \begin{minipage}[b]{\linewidth}\raggedleft
whitespace
\end{minipage} \\
\midrule\noalign{}
\endhead
\bottomrule\noalign{}
\endlastfoot
Gender & 0 & 1 & 4 & 6 & 0 & 2 & 0 \\
EducationLevel & 0 & 1 & 9 & 20 & 0 & 4 & 0 \\
IncomeCategory & 0 & 1 & 10 & 20 & 0 & 5 & 0 \\
\end{longtable}

\textbf{Variable type: numeric}

\begin{longtable}[]{@{}
  >{\raggedright\arraybackslash}p{(\columnwidth - 20\tabcolsep) * \real{0.2720}}
  >{\raggedleft\arraybackslash}p{(\columnwidth - 20\tabcolsep) * \real{0.0800}}
  >{\raggedleft\arraybackslash}p{(\columnwidth - 20\tabcolsep) * \real{0.1120}}
  >{\raggedleft\arraybackslash}p{(\columnwidth - 20\tabcolsep) * \real{0.0880}}
  >{\raggedleft\arraybackslash}p{(\columnwidth - 20\tabcolsep) * \real{0.0880}}
  >{\raggedleft\arraybackslash}p{(\columnwidth - 20\tabcolsep) * \real{0.0480}}
  >{\raggedleft\arraybackslash}p{(\columnwidth - 20\tabcolsep) * \real{0.0720}}
  >{\raggedleft\arraybackslash}p{(\columnwidth - 20\tabcolsep) * \real{0.0560}}
  >{\raggedleft\arraybackslash}p{(\columnwidth - 20\tabcolsep) * \real{0.0720}}
  >{\raggedleft\arraybackslash}p{(\columnwidth - 20\tabcolsep) * \real{0.0640}}
  >{\raggedright\arraybackslash}p{(\columnwidth - 20\tabcolsep) * \real{0.0480}}@{}}
\toprule\noalign{}
\begin{minipage}[b]{\linewidth}\raggedright
skim\_variable
\end{minipage} & \begin{minipage}[b]{\linewidth}\raggedleft
n\_missing
\end{minipage} & \begin{minipage}[b]{\linewidth}\raggedleft
complete\_rate
\end{minipage} & \begin{minipage}[b]{\linewidth}\raggedleft
mean
\end{minipage} & \begin{minipage}[b]{\linewidth}\raggedleft
sd
\end{minipage} & \begin{minipage}[b]{\linewidth}\raggedleft
p0
\end{minipage} & \begin{minipage}[b]{\linewidth}\raggedleft
p25
\end{minipage} & \begin{minipage}[b]{\linewidth}\raggedleft
p50
\end{minipage} & \begin{minipage}[b]{\linewidth}\raggedleft
p75
\end{minipage} & \begin{minipage}[b]{\linewidth}\raggedleft
p100
\end{minipage} & \begin{minipage}[b]{\linewidth}\raggedright
hist
\end{minipage} \\
\midrule\noalign{}
\endhead
\bottomrule\noalign{}
\endlastfoot
Age & 0 & 1 & 40.38 & 13.58 & 19 & 25.0 & 45 & 5.05e+01 & 7.3e+01 &
▇▂▇▅▁ \\
HouseholdIncome & 0 & 1 & 3080671.42 & 6906289.85 & 15000 & 500000.0 &
900000 & 2.00e+06 & 5.0e+07 & ▇▁▁▁▁ \\
Frequency\_LaptopDesktop & 0 & 1 & 3.52 & 1.90 & 0 & 2.0 & 4 & 5.00e+00
& 5.0e+00 & ▃▁▁▃▇ \\
Frequency\_Smartphone & 0 & 1 & 4.22 & 1.57 & 0 & 4.0 & 5 & 5.00e+00 &
5.0e+00 & ▁▁▁▂▇ \\
Frequency\_Tablet & 0 & 1 & 1.94 & 1.83 & 0 & 0.0 & 2 & 3.00e+00 &
5.0e+00 & ▇▂▃▂▂ \\
Frequency\_SmartHomeDevices & 0 & 1 & 2.88 & 1.90 & 0 & 1.0 & 3 &
4.50e+00 & 5.0e+00 & ▇▂▃▆▇ \\
Frequency\_WearableDevices & 0 & 1 & 2.31 & 2.09 & 0 & 0.0 & 2 &
4.50e+00 & 5.0e+00 & ▇▂▁▂▅ \\
Frequency\_GamingConsoles & 0 & 1 & 1.40 & 1.72 & 0 & 0.0 & 0 & 3.00e+00
& 5.0e+00 & ▇▂▂▂▁ \\
Frequency\_HomeSecuritySystem & 0 & 1 & 1.53 & 1.83 & 0 & 0.0 & 1 &
3.00e+00 & 5.0e+00 & ▇▂▂▁▂ \\
LikelinessToAdopt & 0 & 1 & 4.01 & 1.03 & 1 & 3.0 & 4 & 5.00e+00 &
5.0e+00 & ▁▁▃▇▇ \\
Benefit\_ProductivityEfficiency & 0 & 1 & 3.83 & 1.07 & 1 & 3.0 & 4 &
5.00e+00 & 5.0e+00 & ▁▁▃▇▅ \\
Benefit\_CommunicationConnectivity & 0 & 1 & 4.06 & 1.04 & 1 & 4.0 & 4 &
5.00e+00 & 5.0e+00 & ▁▁▂▇▇ \\
Benefit\_EasyAccessToInfo & 0 & 1 & 4.24 & 1.08 & 1 & 4.0 & 5 & 5.00e+00
& 5.0e+00 & ▁▁▂▅▇ \\
Benefit\_Entertainment & 0 & 1 & 4.10 & 1.01 & 1 & 4.0 & 4 & 5.00e+00 &
5.0e+00 & ▁▁▃▇▇ \\
Risk\_SecurityPrivacy & 0 & 1 & 3.82 & 1.11 & 1 & 3.0 & 4 & 5.00e+00 &
5.0e+00 & ▁▂▃▇▆ \\
Risk\_ExcessScreenTime & 0 & 1 & 3.83 & 1.07 & 1 & 3.0 & 4 & 5.00e+00 &
5.0e+00 & ▁▂▃▇▆ \\
Risk\_ComplicatedToUse & 0 & 1 & 2.82 & 1.15 & 1 & 2.0 & 3 & 4.00e+00 &
5.0e+00 & ▃▇▇▅▂ \\
Risk\_Addiction & 0 & 1 & 3.42 & 1.23 & 1 & 3.0 & 3 & 4.00e+00 & 5.0e+00
& ▂▅▇▇▇ \\
Risk\_Health & 0 & 1 & 3.47 & 1.23 & 1 & 3.0 & 4 & 4.50e+00 & 5.0e+00 &
▂▅▇▇▇ \\
Risk\_LossOfConnection & 0 & 1 & 3.68 & 1.18 & 1 & 3.0 & 4 & 5.00e+00 &
5.0e+00 & ▂▂▆▇▇ \\
Influence\_FamilyFriends & 0 & 1 & 3.23 & 1.17 & 1 & 3.0 & 3 & 4.00e+00
& 5.0e+00 & ▂▃▇▇▃ \\
Influence\_PromoActivities & 0 & 1 & 3.06 & 1.11 & 1 & 2.0 & 3 &
4.00e+00 & 5.0e+00 & ▂▃▇▅▂ \\
Influence\_SocialPopularity & 0 & 1 & 3.13 & 1.17 & 1 & 2.0 & 3 &
4.00e+00 & 5.0e+00 & ▂▅▇▇▃ \\
Influence\_ExpertRecommendation & 0 & 1 & 3.50 & 1.16 & 1 & 3.0 & 4 &
4.00e+00 & 5.0e+00 & ▂▂▅▇▃ \\
Influence\_Endorsements & 0 & 1 & 2.39 & 1.22 & 1 & 1.0 & 2 & 3.00e+00 &
5.0e+00 & ▇▇▅▅▂ \\
Barrier\_HighPrice & 0 & 1 & 3.35 & 1.34 & 1 & 2.0 & 3 & 5.00e+00 &
5.0e+00 & ▃▅▇▆▇ \\
Barrier\_TechKnowledge & 0 & 1 & 2.55 & 1.28 & 1 & 1.0 & 3 & 3.00e+00 &
5.0e+00 & ▇▆▇▃▂ \\
Barrier\_PrivacySecurity & 0 & 1 & 3.40 & 1.33 & 1 & 2.0 & 4 & 5.00e+00
& 5.0e+00 & ▃▅▇▇▇ \\
Barrier\_ReliableInternet & 0 & 1 & 2.80 & 1.34 & 1 & 2.0 & 3 & 4.00e+00
& 5.0e+00 & ▇▆▇▇▃ \\
Barrier\_ChangeReluctance & 0 & 1 & 2.42 & 1.16 & 1 & 1.0 & 2 & 3.00e+00
& 5.0e+00 & ▇▇▇▅▁ \\
LikelinessToRecommend & 0 & 1 & 3.90 & 1.14 & 1 & 3.0 & 4 & 5.00e+00 &
5.0e+00 & ▁▁▅▇▇ \\
TotalTechConcern & 0 & 1 & 21.05 & 5.59 & 6 & 18.0 & 22 & 2.50e+01 &
3.0e+01 & ▁▂▅▇▃ \\
TotalTechBenefit & 0 & 1 & 16.23 & 3.76 & 4 & 15.5 & 17 & 1.90e+01 &
2.0e+01 & ▁▁▂▃▇ \\
\end{longtable}

107 rows, 36 columns Data is 100\% complete, no concern over missing
values

\hypertarget{look-at-the-top-and-the-bottom-of-the-data}{%
\subsubsection{Look at the top and the bottom of the
data}\label{look-at-the-top-and-the-bottom-of-the-data}}

\begin{Shaded}
\begin{Highlighting}[]
\FunctionTok{library}\NormalTok{(skimr)}

\ControlFlowTok{if}\NormalTok{ (}\FunctionTok{interactive}\NormalTok{()) \{}
  \FunctionTok{View}\NormalTok{(}\FunctionTok{head}\NormalTok{(SurveyData))}
  \FunctionTok{View}\NormalTok{(}\FunctionTok{tail}\NormalTok{(SurveyData))}
  \FunctionTok{View}\NormalTok{(SurveyData)}
\NormalTok{\}}
\end{Highlighting}
\end{Shaded}

\hypertarget{created-variables-and-relevant-overview-plots}{%
\subsubsection{Created Variables and Relevant Overview
Plots}\label{created-variables-and-relevant-overview-plots}}

\begin{Shaded}
\begin{Highlighting}[]
\FunctionTok{library}\NormalTok{(ggplot2)}

\CommentTok{\# Age distribution Histogram}
\FunctionTok{ggplot}\NormalTok{(SurveyData, }\FunctionTok{aes}\NormalTok{(}\AttributeTok{x =}\NormalTok{ Age)) }\SpecialCharTok{+}
  \FunctionTok{geom\_histogram}\NormalTok{(}
    \AttributeTok{bins =} \DecValTok{20}\NormalTok{,}
    \AttributeTok{fill =} \StringTok{"blue"}\NormalTok{,}
    \AttributeTok{color =} \StringTok{"black"}\NormalTok{,}
    \AttributeTok{alpha =} \FloatTok{0.7}
\NormalTok{  ) }\SpecialCharTok{+}
  \FunctionTok{labs}\NormalTok{(}\AttributeTok{x =} \StringTok{"Age"}\NormalTok{, }\AttributeTok{y =} \StringTok{"Count"}\NormalTok{, }\AttributeTok{title =} \StringTok{"Fig. 1: Distribution of Respondents\textquotesingle{} Age"}\NormalTok{) }\SpecialCharTok{+}
  \FunctionTok{theme\_minimal}\NormalTok{()}
\end{Highlighting}
\end{Shaded}

\includegraphics{DataScienceProjectEDA_files/figure-latex/unnamed-chunk-5-1.pdf}

\begin{Shaded}
\begin{Highlighting}[]
\FunctionTok{library}\NormalTok{(ggplot2)}
\FunctionTok{library}\NormalTok{(dplyr)}
\end{Highlighting}
\end{Shaded}

\begin{verbatim}
## 
## Attaching package: 'dplyr'
\end{verbatim}

\begin{verbatim}
## The following objects are masked from 'package:stats':
## 
##     filter, lag
\end{verbatim}

\begin{verbatim}
## The following objects are masked from 'package:base':
## 
##     intersect, setdiff, setequal, union
\end{verbatim}

\begin{Shaded}
\begin{Highlighting}[]
\CommentTok{\# Gender distribution pie chart}
\NormalTok{SurveyData }\SpecialCharTok{\%\textgreater{}\%}
  \FunctionTok{count}\NormalTok{(Gender) }\SpecialCharTok{\%\textgreater{}\%}
  \FunctionTok{ggplot}\NormalTok{(}\FunctionTok{aes}\NormalTok{(}\AttributeTok{x =} \StringTok{""}\NormalTok{, }\AttributeTok{y =}\NormalTok{ n, }\AttributeTok{fill =}\NormalTok{ Gender)) }\SpecialCharTok{+}
  \FunctionTok{geom\_bar}\NormalTok{(}\AttributeTok{stat =} \StringTok{"identity"}\NormalTok{, }\AttributeTok{width =} \DecValTok{1}\NormalTok{) }\SpecialCharTok{+}
  \FunctionTok{coord\_polar}\NormalTok{(}\StringTok{"y"}\NormalTok{, }\AttributeTok{start =} \DecValTok{0}\NormalTok{) }\SpecialCharTok{+}
  \FunctionTok{scale\_fill\_brewer}\NormalTok{(}\AttributeTok{palette =} \StringTok{"Pastel1"}\NormalTok{) }\SpecialCharTok{+}
  \FunctionTok{theme\_void}\NormalTok{() }\SpecialCharTok{+}
  \FunctionTok{labs}\NormalTok{(}\AttributeTok{title =} \StringTok{"Fig. 2: Gender Distribution"}\NormalTok{) }\SpecialCharTok{+}
  \FunctionTok{geom\_text}\NormalTok{(}\FunctionTok{aes}\NormalTok{(}\AttributeTok{label =}\NormalTok{ scales}\SpecialCharTok{::}\FunctionTok{percent}\NormalTok{(n }\SpecialCharTok{/} \FunctionTok{sum}\NormalTok{(n), }\AttributeTok{accuracy =} \FloatTok{0.1}\NormalTok{)),}
    \AttributeTok{position =} \FunctionTok{position\_stack}\NormalTok{(}\AttributeTok{vjust =} \FloatTok{0.5}\NormalTok{)}
\NormalTok{  )}
\end{Highlighting}
\end{Shaded}

\includegraphics{DataScienceProjectEDA_files/figure-latex/unnamed-chunk-6-1.pdf}

\begin{Shaded}
\begin{Highlighting}[]
\CommentTok{\# Calculating mean and median age}
\NormalTok{mean\_age }\OtherTok{\textless{}{-}}
  \FunctionTok{mean}\NormalTok{(SurveyData}\SpecialCharTok{$}\NormalTok{Age, }\AttributeTok{na.rm =} \ConstantTok{TRUE}\NormalTok{) }\CommentTok{\# na.rm = TRUE ensures that NA values are removed}
\NormalTok{median\_age }\OtherTok{\textless{}{-}} \FunctionTok{median}\NormalTok{(SurveyData}\SpecialCharTok{$}\NormalTok{Age, }\AttributeTok{na.rm =} \ConstantTok{TRUE}\NormalTok{)}

\FunctionTok{cat}\NormalTok{(}\FunctionTok{sprintf}\NormalTok{(}\StringTok{"Mean Age: \%.2f, Median Age: \%.2f"}\NormalTok{, mean\_age, median\_age))}
\end{Highlighting}
\end{Shaded}

\begin{verbatim}
## Mean Age: 40.38, Median Age: 45.00
\end{verbatim}

\begin{Shaded}
\begin{Highlighting}[]
\FunctionTok{library}\NormalTok{(ggplot2)}

\CommentTok{\# Histogram plot of Age with color based on Gender}
\NormalTok{SurveyData}\SpecialCharTok{$}\NormalTok{Gender }\OtherTok{\textless{}{-}} \FunctionTok{factor}\NormalTok{(SurveyData}\SpecialCharTok{$}\NormalTok{Gender, }\AttributeTok{levels =} \FunctionTok{c}\NormalTok{(}\StringTok{"Male"}\NormalTok{, }\StringTok{"Female"}\NormalTok{))}

\FunctionTok{ggplot}\NormalTok{(}\AttributeTok{data =}\NormalTok{ SurveyData, }\FunctionTok{aes}\NormalTok{(}\AttributeTok{x =}\NormalTok{ Age, }\AttributeTok{fill =}\NormalTok{ Gender)) }\SpecialCharTok{+}
  \FunctionTok{geom\_histogram}\NormalTok{(}
    \AttributeTok{position =} \StringTok{"stack"}\NormalTok{,}
    \AttributeTok{binwidth =} \DecValTok{3}\NormalTok{,}
    \AttributeTok{colour =} \StringTok{"grey"}
\NormalTok{  ) }\SpecialCharTok{+}
  \FunctionTok{labs}\NormalTok{(}\AttributeTok{title =} \StringTok{"Fig. 3: Number of Male and Female by Age"}\NormalTok{) }\SpecialCharTok{+}
  \FunctionTok{theme\_minimal}\NormalTok{()}
\end{Highlighting}
\end{Shaded}

\includegraphics{DataScienceProjectEDA_files/figure-latex/unnamed-chunk-8-1.pdf}
Age and gender distribution of participants. The analysis reveals that
the majority of participants identified as male, accounting for 75.7\%
of the sample. The remaining 24.3\% of the participants identified as
female. Furthermore, the analysis reveals a diverse range of ages, with
peaks around 20-25 and 45-50

\hypertarget{income-category-variable-created}{%
\subsubsection{Income Category Variable
Created}\label{income-category-variable-created}}

\begin{Shaded}
\begin{Highlighting}[]
\CommentTok{\# Define labels for the income categories}
\NormalTok{breaks }\OtherTok{\textless{}{-}} \FunctionTok{c}\NormalTok{(}\DecValTok{0}\NormalTok{, }\DecValTok{200000}\NormalTok{, }\DecValTok{600000}\NormalTok{, }\DecValTok{1200000}\NormalTok{, }\DecValTok{2000000}\NormalTok{, }\ConstantTok{Inf}\NormalTok{)}
\NormalTok{labels }\OtherTok{\textless{}{-}}
  \FunctionTok{c}\NormalTok{(}
    \StringTok{"Low Income"}\NormalTok{,}
    \StringTok{"Lower{-}Middle Income"}\NormalTok{,}
    \StringTok{"Middle{-}Middle Income"}\NormalTok{,}
    \StringTok{"Upper{-}Middle Income"}\NormalTok{,}
    \StringTok{"High Income"}
\NormalTok{  )}

\CommentTok{\# Create the IncomeCategory variable}
\NormalTok{SurveyData}\SpecialCharTok{$}\NormalTok{IncomeCategory }\OtherTok{\textless{}{-}}
  \FunctionTok{cut}\NormalTok{(}
\NormalTok{    SurveyData}\SpecialCharTok{$}\NormalTok{HouseholdIncome,}
    \AttributeTok{breaks =}\NormalTok{ breaks,}
    \AttributeTok{labels =}\NormalTok{ labels,}
    \AttributeTok{include.lowest =} \ConstantTok{TRUE}
\NormalTok{  )}
\end{Highlighting}
\end{Shaded}

\hypertarget{variable-for-totaltechconcern-created}{%
\subsubsection{Variable for TotalTechConcern
Created}\label{variable-for-totaltechconcern-created}}

\begin{Shaded}
\begin{Highlighting}[]
\CommentTok{\# TotalTechConcern calculated by summing all columns that start with \textquotesingle{}Risk\textquotesingle{}}
\NormalTok{SurveyData }\OtherTok{\textless{}{-}}\NormalTok{ SurveyData }\SpecialCharTok{\%\textgreater{}\%}
  \FunctionTok{mutate}\NormalTok{(}\AttributeTok{TotalTechConcern =} \FunctionTok{rowSums}\NormalTok{(}\FunctionTok{select}\NormalTok{(., }\FunctionTok{starts\_with}\NormalTok{(}\StringTok{"Risk"}\NormalTok{))))}
\end{Highlighting}
\end{Shaded}

\hypertarget{variable-for-totaltechbenefit-created}{%
\subsubsection{Variable for TotalTechBenefit
Created}\label{variable-for-totaltechbenefit-created}}

\begin{Shaded}
\begin{Highlighting}[]
\CommentTok{\# Calculate the TotalTechBenefit by summing all columns that start with \textquotesingle{}Benefit\textquotesingle{}}
\NormalTok{SurveyData }\OtherTok{\textless{}{-}}\NormalTok{ SurveyData }\SpecialCharTok{\%\textgreater{}\%}
  \FunctionTok{mutate}\NormalTok{(}\AttributeTok{TotalTechBenefit =} \FunctionTok{rowSums}\NormalTok{(}\FunctionTok{select}\NormalTok{(., }\FunctionTok{starts\_with}\NormalTok{(}\StringTok{"Benefit"}\NormalTok{)), }\AttributeTok{na.rm =} \ConstantTok{TRUE}\NormalTok{))}
\end{Highlighting}
\end{Shaded}

\hypertarget{plots-that-look-at-the-questions}{%
\subsubsection{Plots that look at the
questions}\label{plots-that-look-at-the-questions}}

Plotting the correlation between Age and TotalTechConcern Relevant
Question: Does age-group affect appraisals of technology?

\begin{Shaded}
\begin{Highlighting}[]
\FunctionTok{library}\NormalTok{(ggplot2)}

\FunctionTok{ggplot}\NormalTok{(}\AttributeTok{data =}\NormalTok{ SurveyData, }\FunctionTok{aes}\NormalTok{(}\AttributeTok{x =}\NormalTok{ Age, }\AttributeTok{y =}\NormalTok{ TotalTechConcern)) }\SpecialCharTok{+}
  \FunctionTok{geom\_point}\NormalTok{(}\AttributeTok{alpha =} \FloatTok{0.6}\NormalTok{) }\SpecialCharTok{+}
  \FunctionTok{geom\_smooth}\NormalTok{(}\AttributeTok{method =} \StringTok{"lm"}\NormalTok{, }\AttributeTok{color =} \StringTok{"blue"}\NormalTok{) }\SpecialCharTok{+}
  \FunctionTok{labs}\NormalTok{(}
    \AttributeTok{x =} \StringTok{"Age"}\NormalTok{, }\AttributeTok{y =} \StringTok{"Total Tech Concern"}\NormalTok{,}
    \AttributeTok{title =} \StringTok{"Fig. 4: Correlation between Age and Total Tech Concern"}
\NormalTok{  ) }\SpecialCharTok{+}
  \FunctionTok{theme\_minimal}\NormalTok{()}
\end{Highlighting}
\end{Shaded}

\begin{verbatim}
## `geom_smooth()` using formula = 'y ~ x'
\end{verbatim}

\includegraphics{DataScienceProjectEDA_files/figure-latex/unnamed-chunk-12-1.pdf}

\begin{Shaded}
\begin{Highlighting}[]
\CommentTok{\# Calculate and print the correlation coefficient}
\NormalTok{correlation }\OtherTok{\textless{}{-}}
  \FunctionTok{cor.test}\NormalTok{(SurveyData}\SpecialCharTok{$}\NormalTok{Age, SurveyData}\SpecialCharTok{$}\NormalTok{TotalTechConcern, }\AttributeTok{use =} \StringTok{"complete.obs"}\NormalTok{) }\CommentTok{\# use handles missing values}
\FunctionTok{print}\NormalTok{(correlation)}
\end{Highlighting}
\end{Shaded}

\begin{verbatim}
## 
##  Pearson's product-moment correlation
## 
## data:  SurveyData$Age and SurveyData$TotalTechConcern
## t = 0.78713, df = 105, p-value = 0.433
## alternative hypothesis: true correlation is not equal to 0
## 95 percent confidence interval:
##  -0.1149389  0.2626302
## sample estimates:
##        cor 
## 0.07659085
\end{verbatim}

Correlation between age and technology concern. Total Technology Concern
is calculated by summing up all variables associated with questions
about technology risk. Specifically, It represents the total concern
score for each respondent, considering multiple risks associated with
technology adoption

\hypertarget{plotting-the-relationship-between-incomecategory-and-totaltechconcern}{%
\subsubsection{Plotting the relationship between IncomeCategory and
TotalTechConcern}\label{plotting-the-relationship-between-incomecategory-and-totaltechconcern}}

Relevant Question: Whether income-level negatively affects an
individuals appraisals of technology (lower = more neg)

\begin{Shaded}
\begin{Highlighting}[]
\FunctionTok{library}\NormalTok{(ggplot2)}

\FunctionTok{ggplot}\NormalTok{(}\AttributeTok{data =}\NormalTok{ SurveyData, }\FunctionTok{aes}\NormalTok{(}\AttributeTok{x =}\NormalTok{ IncomeCategory, }\AttributeTok{y =}\NormalTok{ TotalTechConcern)) }\SpecialCharTok{+}
  \FunctionTok{geom\_boxplot}\NormalTok{() }\SpecialCharTok{+}
  \FunctionTok{labs}\NormalTok{(}
    \AttributeTok{x =} \StringTok{"Income Category"}\NormalTok{, }\AttributeTok{y =} \StringTok{"Total Tech Concern"}\NormalTok{,}
    \AttributeTok{title =} \StringTok{"Fig. 5: Total Tech Concern across Income Categories"}
\NormalTok{  ) }\SpecialCharTok{+}
  \FunctionTok{theme\_minimal}\NormalTok{()}
\end{Highlighting}
\end{Shaded}

\includegraphics{DataScienceProjectEDA_files/figure-latex/unnamed-chunk-13-1.pdf}

\begin{Shaded}
\begin{Highlighting}[]
\CommentTok{\# ANOVA to test if differences in TotalTechConcern are statistically significant across IncomeCategory}
\NormalTok{anova\_result }\OtherTok{\textless{}{-}} \FunctionTok{aov}\NormalTok{(TotalTechConcern }\SpecialCharTok{\textasciitilde{}}\NormalTok{ IncomeCategory, }\AttributeTok{data =}\NormalTok{ SurveyData)}
\FunctionTok{summary}\NormalTok{(anova\_result)}
\end{Highlighting}
\end{Shaded}

\begin{verbatim}
##                 Df Sum Sq Mean Sq F value Pr(>F)
## IncomeCategory   4    100   24.92   0.792  0.533
## Residuals      102   3211   31.48
\end{verbatim}

Relationship between income and technology concerns. Statistical
analysis suggests no significant difference in TotalTechConcern among
the income groups (p \textgreater{} 0.05). Income values are divided
into five categories: Low Income (0 to 200,000), Lower-Middle Income
(200,001 to 600,000), Middle-Middle Income (600,001 to 1,200,000), Upper
Middle (1,200,001 to 2,000,000) Income, and High Income (Above
2,000,000). Each respondent is assigned an income category label based
on their household income range.

\hypertarget{discussion}{%
\section{Discussion}\label{discussion}}

P \textgreater{} 0.05 (No significant effect) for both observations.
Therefore, income-level and age-group does not correlate with an
individuals appraisals of technology.

Problem of Gender Imbalance: To counter the overrepresentation of male
participants, future iterations of the survey could aim for a more
balanced gender representation.

Diversifying Geographic Representation: The data is gathered from
participants residing in India. As a result, generalizability is called
into question. Conducting the survey in different countries or regions
would provide comparative insights and reveal whether the trends
observed are specific to India or more universally applicable.

Fit for Purpose: The data aligns well with the specific research
questions I have, however, the sample representativeness is a weak point
of the data. generalizability is not possible with this data, therefore,
it is not fit for purpose. Regardless, the data provides a good starting
point.

\end{document}
